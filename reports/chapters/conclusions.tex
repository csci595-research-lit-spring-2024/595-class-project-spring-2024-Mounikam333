\chapter{Conclusions and Future Work}
\label{ch:con}
\section{Conclusion}

In conclusion, this study highlights the effectiveness of regression models in forest fire prediction. Through the implementation of appropriate preprocessing techniques and model selection strategies, significant enhancements in predictive accuracy can be achieved. Among the various models evaluated, CatBoost emerged as the top performer. Notably, CatBoost possesses unique capabilities in handling diverse data types and detecting complex patterns, contributing to its superior performance compared to other models.

CatBoost's robustness and adaptability make it a valuable tool for forest fire prediction and management. Its ability to handle different types of data with ease and discern intricate relationships within the data sets it apart from other models. Leveraging the strengths of CatBoost and similar advanced techniques will be crucial for further improving predictive accuracy and refining forest fire mitigation strategies.the model attained impressive metrics: MSE=1.89, RMSE=1.37, MAE=1.15, and R²=0.0052, marking exceptional performance, as seen in table 4.

Continued research efforts are imperative to address existing limitations and further optimize the application of advanced modeling techniques in forest fire prediction and management. By harnessing the potential of CatBoost and continuously refining modeling approaches, we can bolster our capabilities in predicting and mitigating the impact of forest fires.
