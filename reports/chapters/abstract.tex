%Two resources useful for abstract writing.
% Guidance of how to write an abstract/summary provided by Nature: https://cbs.umn.edu/sites/cbs.umn.edu/files/public/downloads/Annotated_Nature_abstract.pdf %https://writingcenter.gmu.edu/guides/writing-an-abstract
\chapter*{\center \Large  Abstract}


Forest fires pose significant threats to ecosystems, human lives, and infrastructure. Predicting forest fire occurrence is crucial for effective resource allocation, mitigation, and recovery efforts. This paper explores recent advancements in forest fire prediction methodologies, mainly focusing on integrating artificial intelligence (AI) and statistical inference techniques. We discuss the implications of reduced parameter sets in AI-based models for efficient prediction systems, especially pertinent to developing countries. Moreover, we delve into the statistical properties of random forest models, shedding light on their error distributions and potential for statistical inference. Through a comprehensive literature review and comparative analysis, we aim to provide insights into cutting-edge approaches for forest fire prediction, paving the way for more accurate and reliable prediction systems.


%%%%%%%%%%%%%%%%%%%%%%%%%%%%%%%%%%%%%%%%%%%%%%%%%%%%%%%%%%%%%%%%%%%%%%%%%s
~\\[1cm]
\noindent % 
\textbf{Keywords:} Forest fire occurrence prediction, Support vector machines, Artificial neural networks, Feature Reduction, Weather data







