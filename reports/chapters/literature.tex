\chapter{Literature Review}
\label{ch:lit_rev} %Label of the chapter lit rev. The key ``ch:lit_rev'' can be used with command \ref{ch:lit_rev} to refer this Chapter.
\section{Introduction}
\label{sec:into_back}

Predicting forest fire sizes is essential for implementing effective mitigation strategies and minimizing their destructive impact. In recent years, data mining techniques and meteorological data analysis have emerged as promising approaches for forest fire prediction. This literature survey examines notable studies in this domain, which propose various data-driven and climate-based models for predicting forest fire sizes. Through a comprehensive analysis, this survey aims to provide insights into the methodologies employed, their strengths and limitations, and opportunities for further research to enhance forest fire prediction accuracy and facilitate proactive management and mitigation efforts.

\section{Example of in-text citation of references in LaTeX
}
\label{sec:into_back}
A study found that 21.9% of wildfire firefighter deaths from 1990-2006 were due to heart attacks.Mangan (2007)
Different scales are used for each of the FWI elements, high values suggest more severe burning conditions (Taylor and Alexander 2006)

\section{Example of "risk" of unintentional plagiarism
}
\label{sec:into_back} 
Unintentional plagiarism arises when writers neglect to properly acknowledge borrowed information, often due to oversight. One common scenario involves the omission of citations for widely recognized facts or common knowledge within a specific field or context. 
For example, failing to attribute the fact that "water boils at 100 degrees Celsius at sea level" can inadvertently lead to plagiarism, even though it is widely acknowledged. This oversight, particularly in academic or formal writing, underscores the importance of diligently crediting all sources to maintain integrity and avoid unintentional plagiarism.

\section{Critique of the review}
\label{sec:into_back} 
The review provides an extensive analysis of methodologies employed in forest fire prediction, spanning data mining techniques, meteorological variables, and machine learning algorithms. [6] notably focused on investigating various data mining techniques, particularly Support Vector Machines (SVM), to forecast forest fire sizes. While their study highlighted the effectiveness of SVM, a deeper critique is warranted regarding the challenges associated with implementing these techniques, including data availability, model complexity, and computational requirements . [7] hybrid model, integrates clustering and classification techniques, presents promising outcomes in forest fire prediction. Their approach, while innovative, lacks a comparative analysis with existing methodologies to fully elucidate its strengths and weaknesses. Furthermore, the review overlooks external factors like climate change and land-use patterns, which could significantly impact predictive accuracy
Additionally [8] explores on the application of Random
Forests, emphasizing ensemble methods' potential in capturing intricate relationships between meteorological variables and
fire occurrence. While their study offers valuable insights, 
a deeper examination of the interpretability and robustness 
of Random Forest models is needed. Furthermore, discussing 
the scalability of these algorithms and their suitability for
real-time prediction in large-scale forest areas would 
provide practical implications for forest fire management. A 
research on the influence of climate change on forest fire 
regimes underscores the importance of incorporating climate 
projections into predictive models[9] . Their emphasis on 
considering long-term trends and variability in climate parameters is noteworthy. However, the review could elaborate 
on the specific methodologies proposed for integrating 
climate data into predictive models and discuss challenges
related to climate model uncertainty and down scaling 
techniques. Additionally, exploring the implications of 
changing fire weather patterns on forest fire behavior and 
the effectiveness of current mitigation strategies would enrich the discussion and provide valuable insights for
future research.

\section{Summary}
\label{sec:into_back}
The exploration of methodologies for forest fire prediction reveals promising avenues through data mining techniques, 
meteorological variables, and machine learning algorithms. 
While studies showcase the effectiveness of Support Vector Machines (SVM), hybrid models, and Random Forests, there 
remains a need for a deeper critique of their limitations and 
challenges, including data availability, model complexity, 
and scalability. Moreover, the significance of considering 
external factors like climate change and land-use patterns is
evident, urging the integration of climate projections into 
predictive models. Addressing these aspects will be pivotal 
in advancing forest fire management and mitigation strategies.









