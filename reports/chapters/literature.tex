\chapter{Literature Review}
\label{ch:lit_rev} %Label of the chapter lit rev. The key ``ch:lit_rev'' can be used with command \ref{ch:lit_rev} to refer this Chapter.
\section{Introduction}
\label{sec:into_back}

The possibility of continuous cyberattacks has complemented a tenacious and broadening issue in 
the current steady exchange of computerized space. Getting computerized homes gives prominent 
disadvantages, as fathomed by the ongoing extension in the figure and perfection of these 
attacks. Accordingly, harmful entertainers' strategies progress going with the contraption, it 
is imperious to take obstructive contemplation to consolidate susceptibilities. By avoiding and 
lessening these steady exchanging dangers, this proposition started an escalated study, giving 
the convolutions of cyberattacks. The essential focal point of the section is to feature and 
assess the most dependable study in light of the examination subject of the adequacy of the 
latest things
and techniques utilized in digital assault and the association's advancement to forestall the 
digital assault that considers the dynamic nature of cyber-security. This section has involved 
ideas
regarding characterizing the meaning of the examination subject. In this way, the literature 
has fostered its key idea in light of applicable studies given the referenced exploration 
subject as far


\section{Discussion on the Current Trends and Methods used in Cyber-Attack}
\label{sec:into_back}
 
The dynamic landscape of cyber security has created a battleground for hackers and security 
providers. With the advancement of technology, the methods are upgraded by malicious hackers to 
exploit the vulnerability. The current trends and methods in the cyber attack will be determined
below:
\begin{itemize}
     \item  Remote working cyber security risk

     The remote work generated by the COVID-19 pandemic opened up new cyber issues for 
     organizations as well as individuals. Remote officers are often less fortified than the 
     centralized office, which becomes a breeding ground for  cyber hackers (Kaspersky, 2019). 
     The use of personal devices that link professional life and personal life impacts the risk 
     of sensitive information falling into the wrong hands. Organizations need to develop and 
     identify security vulnerabilities by improving their system of implementing control that 
     ensures proper monitoring for a secure distributed workforce.
    \item IoT evolution 
    The Internet of Things (IoT) is growing, providing cybercriminals with a larger attack 
    surface. IoT devices, fluctuating from wearables to keen home strategies, repeatedly lack 
    rigorous security actions (Kaspersky, 2019). This absence generates challenges in retaining 
    outdated security claims, making IoT a well-paid target. As the quantity  of IoT strategies 
    is predicted to reach 64 billion worldwide by 2026, administrations must augment their 
    safety posture to precaution in contradiction of latent occurrences.

	\item The rise of Ransomware A persistent threat, Ransomware, has been a surge in 
 sophistication with frequency. 
    The improving digitization landscape during the time of pandemic has upgraded the remote 
    work that provided cyber criminals with their target (Kaur and Kumar K.R, 2021). The 
    attacks involve encryption of companies' data that provides a threat for releasing 
    sensitive information. The financial and reputation consequences of the attacks have been 
    identified as creating a concern for the companies.
    
    \item Cloud services with security threats the boundless reception of cloud administrations 
    has reformed how organizations work; however, it has likewise become an ideal objective for 
    digital aggressors (Kaur and Kumar K.R, 2021). Misconfigured cloud settings, uncertain 
    Points of interaction, and record seizing present serious dangers. According to Cabaj et al.
    (2018), associations should address difficulties connected with administrative consistency, 
    IT aptitude, and potential section focuses on assailants, building up their cloud safety 
    efforts.


	 \item Smarter social engineering attack '
    The engineering attacks have become more sophisticated for cyber hackers, as they target 
    remote workers. The most innovative social attacks focused on the leadership of the 
    organization, smashing and vising. Cyber hackers constantly
    improve their platforms in innovative ways, like messaging apps, to trick users.
	\item Data privacy
    The high-profile data breaches with the data protection law have elevated the 
    organization's privacy.However,noncompliance risks reputational damage and loss of trust 
    can impact the organization created by cyber hackers. As per the view of Cabaj et al. 
    (2018), organizations are focused on data privacy by appointing data privacy officers, 
    implementing encryption, and undergoing external assessment.
	\item Multi-factor authentication improvement
    While MFA is viewed as a strong verification strategy, aggressors track down ways of 
    bypassing it, primarily through SMS or telephone-based confirmation. Associations are 
    moving towards application-based MFA to address weaknesses related to SMS validation. To 
    stay ahead of malicious actors, MFA methods must continue to evolve.
	\item Rise of AI tools
    The cyber threat has led organizations to leverage AI and machine learning for analyzing 
    security infrastructure. AI aids in the automatic security system for threat detection and 
    data analysis for human capability at a pace beyond the capability of humans.As per the 
    statement of Cabaj et al. (2018), the automatic attacks for AI tools have been developed by 
    data-driven security tools of cyber hackers.
	\item Mobile cyber security
    With the high usage of remote working mobile devices have become
   more Central for everyday operation. The ever-trending usage of mobile has improved the 
   mobile threads that include spyware security vulnerabilities with the malware software of 
   mobile. As the 5G technology rollout, organisations are impacted by more cyber hackers. The 
   cyber threats in the dynamic technology progress have been updated due to the 5G network 
   progress.
\end{itemize}


\section{Identification of the Organization's Development to Prevent the Cyber-Attack}
\label{sec:into_back} 
Protecting against cyber-attacks requires incorporating policies and education technology with 
ongoing development. The organizations that face issues with cyber security
need to develop these measures in the dynamic nature of the cyber threat:
\begin{itemize}
    \item Continuous network and database security
Safeguard the organizations by setting up firewalls and scrambling data. This will assist with 
limiting the gamble of digital hoodlums accessing secret data. Ensure the Wi-Fi network is 
covered up and the secret phrase is safeguarded. Be careful when selecting the data that is 
saved in the company databases. As stated by Mass (2023), data sets can be an incredible means 
for organizations to have a focal area of information and reports, yet this doesn't mean 
putting away all information is positive. Programmed upholding of organization information 
ought to be set to be finished either one time each day or one time each week, contingent upon 
the degree of action inside the organization. Backing up the organization's information will 
improve the probability of a digital assault. Hence, the organization's information won't be 
lost totally, which is very much standard.

    \item Employee education with training
Educating the employees is an aspect of strengthening the cyber security posture for the 
company. On the other hand, the conduction of training programs can also raise awareness of the 
importance of cyber security in the organization. As opined by Mass (2023), transparent 
communication systems with employees provide safeguards for the company that also provides 
security for customer data and colleagues' details. Establishing enforcement policies 
delineating acceptable practices also leaves it the axis for minimizing the risk of downloading 
malicious software. 
    \item A security policy with practices
The comprehensive security policies in the organization are tailored to the practice with specific needs. These guidelines cover several aspects, including data protection, data security, and network security with incident response. By clearly outlining the procedure for the following in the event of a security bridge, the policy violation also decreases (Mohamed, 2023). However, controlling physical access to the company ensures the proper disposable procedure. It also has to prevent authorized access to the computer system or handling devices that reduce the likelihood of cyber threats. 
     \item Awareness of fake antivirus offers
Employee training helps to recognize and distinguish them from fake antivirus offers and notifications. However, by developing a clear policy for reporting suspicious activity the cyber hackers often used the tactic to track the users for downloading malicious software (Mohamed, 2023). Establishing a protocol for handling the infected computers emphasizes the importance of reporting to the IT department. Additionally, regular updates to the security software provide safeguards against the cyber security threat.
    \item Customer communication protection
Building trust and ensuring the security of customers' personal information require open lines of communication—expressive explanations behind gathering client information and how it will be utilized. The narrator of Cs et al. (2017) guarantees clients that the association won't ever demand touchy data through unprotected correspondence channels, such as email or instant messages. Urge clients to report dubious correspondences, cultivating a cooperative way to deal with online protection.
    \item Dynamic organization development
Companies must impress the adaptive approaches of cyber security to prevent cyber attacks. It includes ongoing assistance in enhancing the security measures in responding to emerging threats. However, implementing regular risk assessment with the scan of vulnerability can find the potential weaknesses in the security infrastructure. For instance, key-logging malware can follow all that the client types on their console. This implies digital lawbreakers could access financial balances, client data, passwords, and other organisation delicate data. As narrated by Cs et al. (2017), to stay up with the latest to help forestall malware from sneaking into your framework and organizations. On the other hand, a faster culture for continuous improvement also provides cyber security practices that update with the evolving threat of the landscape. 
    \item Collaboration and information sharing
Collaboration with industrial information sharing is an essential component of cyber security. By engaging with the cyber security communities, the organization can identify threat intelligence by sharing initiatives and findings about the latest trends with attack vectors (Rajasekharaiah et al., 2020). On the other hand, collaboration with other fields of organization and business can gain valuable insight into emerging ideas for threats that contribute to a cyber security strategy in a resilient way. 

\end{itemize}

\begin{table}[h!]
    \centering
    \caption{SWOT Analysis for Cybersecurity Countermeasures}
    \label{tab:_ex_tab}
    \begin{tabular}{llr}     
    \toprule
    \cmidrule(r){1-2}
    Strength    &  Weakness   \\
    \midrule
    Firewall protection                &Lack of mobile security   \\
    Regular data backups with security & insufficient employee      awareness \\
    Updated anti-virus software        &Third party authentication \\
    Training programs of employees     &Limited response plan      \\
        \bottomrule
    \end{tabular}
\end{table}

\begin{table}[h!]
    \centering
    \caption{SWOT Analysis for Cybersecurity Countermeasures}
    \label{tab:_ex_tab}
    \begin{tabular}{llr}     
    \toprule
    \cmidrule(r){1-2}
    Opportunities    &  Threats   \\
    \midrule
    Advances in cloud computing   & Evolving Phishing techniques  \\
    Emerging in authentication tech       &Increase Iot Vulnerabilities   \\
    Integration with AI security          &Ransomware attacks     \\
    Collaboration with cyber security community&Regulatory landscape of cyber security      \\
        \bottomrule
    \end{tabular}
\end{table}



\textbf{Analysis}

\textbf{Strength}
A strong firewall has provided a foundation and defense for authorized access. A regular update in optimizing the protection is crucial for preventing a cyber attack. On the other hand, consistent data backup also reduces the impact of ransom attacks and minimizes the nation of the downtown in the event of a breach (Cremer et al., 2022). By keeping up the antivirus software, it has enhanced the ability to detect malware, which reduces the risk of infections. Investing in continuous employee training programs provides identification of potential security threats.

\textbf{Weakness}
The efficient awareness among the employees increases the vulnerability through which the training programs of education are essential for empowering the workforce to address cyber security threats (Cremer et al., 2022). However, a comprehensive incident response plan is crucial for mitigating the impact of cyber attacks. A regular testing plan to enhance cyber security is necessary for Swift and effective response (Radanliev et al., 2020). Falling to ensure the security of third-party vendors can also impact the organization to its security structure. With the high rise of remote work and mobile security is implementing a robust measure for securing company data. 

\textbf{Opportunities}
 Artificial intelligence for threat detection is responsible for helping to enhance the ability of organisations to evolve cyber threats (Radanliev et al., 2020). Engaging in the threat intelligence sharing initiative provides organizations with them to stay focused on emerging threads that help to adopt proactive countermeasures. Implementation of technological developments in cloud security confirms a vigorous and ascendable defense against developing cyber pressures (Altulaihan et al., 2022). Adopting new confirmation means, such as biometrics or multi-factor confirmation, fortifies access joysticks.

 \textbf{Threats}
Threats 
Cybercriminals unceasingly refine phishing methods. Often, bringing up-to-date and underpinning operative training curricula is critical to hostage evolving pressures (Altulaihan et al., 2022). The cumulative cleverness of ransomware poses a noteworthy threat. Regular reviewing and apprising occurrence response strategies are indispensable for modifying the impression of ransomware occurrences. The growing Internet of Things surges the attack shallow (Riggs et al., 2023). The Establishment of IoT security actions is serious about preventing possible breaches. Acquiescence with managerial requirements is motivating but crucial. Solid audits and up-to-date refuge policies are required to pilot a complex monitoring countryside.

\section{Identification of the role that plays in the face of rapidly changing cyber threat in cyber security defense for organization}
\label{sec:into_back}
The Role of cybersecurity Defence for organizations is securing the organization's data from internal and External threats. This involves a wide range of technologies, processes, structures, and practices aimed at safeguarding networks, computers, programs,  and data against unauthorized access or damage (B. Poornima, 2023). At its core the cybersecurity team is tasked with shielding the IT infrastructure from vulnerabilities, implementing security measures,  monitoring suspicious activities, and swiftly responding to the cyber threat. Their collaborative efforts with the other departments aim to enhance the organization's overall security posture. 
The Security Operations Center (SOC) plays a pivotal role in organizational security, with distinct team responsibilities. Security Analysts conduct real-time vulnerability assessments and monitor threat intelligence. Incident Responders excel in immediate response and disaster recovery during security breaches. Security Architects strategically plan, design, and review the overall security posture (Cynet, 2023). This collaborative approach ensures a comprehensive defense against cyber threats, addressing both immediate incidents and fortifying the organization's long-term security resilience.
Cybersecurity professionals, as strategists, proactively implement security measures, considering consequences and evaluating workflows, dependencies, and resources. They stay ahead of evolving hacking methods by studying entry points and countermeasures (Cynet, 2023). As communicators, management and interpersonal skills are crucial for effective coordination. Being lifelong learners, they maintain technical competence through continuous research, training, and certifications, addressing complex security challenges. 

The Important Roles of Cybersecurity :
\begin{itemize}

	\item Leadership Responsibilities of the CISO: the CISO plays a crucial leadership role in the cybersecurity department and provides direction on various programs, including audits and risk management (Book, 2023).
	\item One essential focus is security analysis, involving the gathering of threat intelligence and monitoring the potential vulnerabilities in security systems.
	\item The security engineers construct and maintain the security architecture, ensuring up-to-date endpoint protection and software development (Book, 2023). 
	\item The incident response plays a vital role during a cyber attack by leading the development and execution of incidence response strategies to minimize the damages. 

Cybersecurity departments employ various defensive measures to protect against cyber threats.
Establishing a robust Cybersecurity Awareness Training Program is crucial for organizational resilience. The program should emphasize the significance of a data backup strategy within the broader security framework, stressing its role in disaster recovery and business continuity. Addressing Common Vulnerabilities and Exposures (CVEs) is essential, providing insights into system-specific security risks (Book, 2023). Introducing Extended Detection and Response (XDR) solutions is recommended, offering a proactive approach to detect and respond to unauthorised access and cyber threats comprehensively. Cynet's innovative monitoring of endpoint memory, focusing on identifying exploit-like behavior, further enhances the organization's defense against evolving cyber risks.
\end{itemize}

\section{Theoretical Underpinning}
\label{sec:into_back}

Fortifying Cyber Security: Embracing Theoretical Framework Against Cyber Attack
Welsh Security Theory
The Welsh Security Theory, often referred to as the "Security Onion" model, plays a crucial role in mitigating cyber security threats within organizations. This model advocates for a layered and comprehensive approach to security, establishing a robust defense-in-depth strategy (Ifac, 2023). By incorporating various defense measures such as firewalls, antivirus software, intrusion detection systems, and user training, the Security Onion model creates a multi-faceted security framework. The interconnected layers work together to fortify the organization's defenses, making it more challenging for cyber threats to breach. Furthermore, the model emphasizes continuous monitoring, incident response planning, and collaborative efforts, fostering adaptability and resilience in the face of evolving cyber threats. Implementation of the Welsh Security Theory enhances overall security resilience, reducing vulnerabilities in the dynamic landscape of cyber threats.

\textbf{Copenhagen Security Theory}

The Copenhagen School's securitization theory was initially formulated for traditional security issues and can be adapted to enhance cyber security defense. In the context of cybersecurity, securitization involves framing specific cyber threats as existential risks, gaining acknowledgment from the audience, and allowing for extraordinary measures. By applying securitization theory, organizations can prioritize cyber threats effectively, elevating them above routine concerns (Diskaya, 2013). This facilitates allocating resources, attention, and swift responses to potential cyber risks. The identification of cyber threats as existential can lead to the implementation of robust defense mechanisms and the development of emergency response plans.
Furthermore, the Copenhagen School's emphasis on desecuritization can be valuable. Shifting issues back into the ordinary public sphere means addressing cyber threats within the context of routine practices rather than treating them as exceptional. This approach encourages a continuous, integrated, and normalized cybersecurity strategy, fostering resilience against evolving cyber threats.

\section{Literature gap}
\label{sec:into_back}
The general writing audit part can be characterized to be viable as far as framing the viability of network protection for associations or people in diminishing Digital Assault. In any case, a few huge holes can be characterized in this part connected with the social event of data. The writing papers considered in this section have not had the option to give explicit models concerning the examination subject. Besides, it very well may be featured that the section could incorporate more applicable examination papers in light of the subject. Consequently, these holes in the writing part can be featured which could be dispensed with.

\section{Summary}
\label{sec:into_back}
The literature review offers an inclusive overview of the active scenery of cyber sanctuary, the importance of the current drifts, and approaches employed by cyber aggressors. The escalating risks related to isolated working, IoT fruition, ransomware, cloud amenities, communal commerce, data discretion, multi-factor confirmation, AI, and mobile cybersecurity require administrations to adopt practical and adaptive approaches. The operation of the Welsh Security Theory and Copenhagen Security Theory appears as appreciated frameworks, stressing defense-in-depth and secularization ideologies. The recognized countermeasures, shown through SWOT analysis, recommend a deliberate road map for administrations to increase their cybersecurity carriage. The embryonic role of cybersecurity in the aspect of quickly changing pressures emphasizes the position of control tasks, security analysis, occasion response, and collective efforts. Regardless of the overall success of the literature, there is a gap in as long as specific specimens and more germane research credentials on the preferred exploration topic, portentous avenues for forthcoming investigation.


